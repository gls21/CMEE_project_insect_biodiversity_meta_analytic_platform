% Options for packages loaded elsewhere
\PassOptionsToPackage{unicode}{hyperref}
\PassOptionsToPackage{hyphens}{url}
%
\documentclass[
]{article}
\usepackage{lmodern}
\usepackage{amssymb,amsmath}
\usepackage{ifxetex,ifluatex}
\ifnum 0\ifxetex 1\fi\ifluatex 1\fi=0 % if pdftex
  \usepackage[T1]{fontenc}
  \usepackage[utf8]{inputenc}
  \usepackage{textcomp} % provide euro and other symbols
\else % if luatex or xetex
  \usepackage{unicode-math}
  \defaultfontfeatures{Scale=MatchLowercase}
  \defaultfontfeatures[\rmfamily]{Ligatures=TeX,Scale=1}
\fi
% Use upquote if available, for straight quotes in verbatim environments
\IfFileExists{upquote.sty}{\usepackage{upquote}}{}
\IfFileExists{microtype.sty}{% use microtype if available
  \usepackage[]{microtype}
  \UseMicrotypeSet[protrusion]{basicmath} % disable protrusion for tt fonts
}{}
\makeatletter
\@ifundefined{KOMAClassName}{% if non-KOMA class
  \IfFileExists{parskip.sty}{%
    \usepackage{parskip}
  }{% else
    \setlength{\parindent}{0pt}
    \setlength{\parskip}{6pt plus 2pt minus 1pt}}
}{% if KOMA class
  \KOMAoptions{parskip=half}}
\makeatother
\usepackage{xcolor}
\IfFileExists{xurl.sty}{\usepackage{xurl}}{} % add URL line breaks if available
\IfFileExists{bookmark.sty}{\usepackage{bookmark}}{\usepackage{hyperref}}
\hypersetup{
  pdftitle={tidyverse style guide tutorial},
  pdfauthor={Grace Skinner},
  hidelinks,
  pdfcreator={LaTeX via pandoc}}
\urlstyle{same} % disable monospaced font for URLs
\usepackage[margin=1in]{geometry}
\usepackage{color}
\usepackage{fancyvrb}
\newcommand{\VerbBar}{|}
\newcommand{\VERB}{\Verb[commandchars=\\\{\}]}
\DefineVerbatimEnvironment{Highlighting}{Verbatim}{commandchars=\\\{\}}
% Add ',fontsize=\small' for more characters per line
\usepackage{framed}
\definecolor{shadecolor}{RGB}{248,248,248}
\newenvironment{Shaded}{\begin{snugshade}}{\end{snugshade}}
\newcommand{\AlertTok}[1]{\textcolor[rgb]{0.94,0.16,0.16}{#1}}
\newcommand{\AnnotationTok}[1]{\textcolor[rgb]{0.56,0.35,0.01}{\textbf{\textit{#1}}}}
\newcommand{\AttributeTok}[1]{\textcolor[rgb]{0.77,0.63,0.00}{#1}}
\newcommand{\BaseNTok}[1]{\textcolor[rgb]{0.00,0.00,0.81}{#1}}
\newcommand{\BuiltInTok}[1]{#1}
\newcommand{\CharTok}[1]{\textcolor[rgb]{0.31,0.60,0.02}{#1}}
\newcommand{\CommentTok}[1]{\textcolor[rgb]{0.56,0.35,0.01}{\textit{#1}}}
\newcommand{\CommentVarTok}[1]{\textcolor[rgb]{0.56,0.35,0.01}{\textbf{\textit{#1}}}}
\newcommand{\ConstantTok}[1]{\textcolor[rgb]{0.00,0.00,0.00}{#1}}
\newcommand{\ControlFlowTok}[1]{\textcolor[rgb]{0.13,0.29,0.53}{\textbf{#1}}}
\newcommand{\DataTypeTok}[1]{\textcolor[rgb]{0.13,0.29,0.53}{#1}}
\newcommand{\DecValTok}[1]{\textcolor[rgb]{0.00,0.00,0.81}{#1}}
\newcommand{\DocumentationTok}[1]{\textcolor[rgb]{0.56,0.35,0.01}{\textbf{\textit{#1}}}}
\newcommand{\ErrorTok}[1]{\textcolor[rgb]{0.64,0.00,0.00}{\textbf{#1}}}
\newcommand{\ExtensionTok}[1]{#1}
\newcommand{\FloatTok}[1]{\textcolor[rgb]{0.00,0.00,0.81}{#1}}
\newcommand{\FunctionTok}[1]{\textcolor[rgb]{0.00,0.00,0.00}{#1}}
\newcommand{\ImportTok}[1]{#1}
\newcommand{\InformationTok}[1]{\textcolor[rgb]{0.56,0.35,0.01}{\textbf{\textit{#1}}}}
\newcommand{\KeywordTok}[1]{\textcolor[rgb]{0.13,0.29,0.53}{\textbf{#1}}}
\newcommand{\NormalTok}[1]{#1}
\newcommand{\OperatorTok}[1]{\textcolor[rgb]{0.81,0.36,0.00}{\textbf{#1}}}
\newcommand{\OtherTok}[1]{\textcolor[rgb]{0.56,0.35,0.01}{#1}}
\newcommand{\PreprocessorTok}[1]{\textcolor[rgb]{0.56,0.35,0.01}{\textit{#1}}}
\newcommand{\RegionMarkerTok}[1]{#1}
\newcommand{\SpecialCharTok}[1]{\textcolor[rgb]{0.00,0.00,0.00}{#1}}
\newcommand{\SpecialStringTok}[1]{\textcolor[rgb]{0.31,0.60,0.02}{#1}}
\newcommand{\StringTok}[1]{\textcolor[rgb]{0.31,0.60,0.02}{#1}}
\newcommand{\VariableTok}[1]{\textcolor[rgb]{0.00,0.00,0.00}{#1}}
\newcommand{\VerbatimStringTok}[1]{\textcolor[rgb]{0.31,0.60,0.02}{#1}}
\newcommand{\WarningTok}[1]{\textcolor[rgb]{0.56,0.35,0.01}{\textbf{\textit{#1}}}}
\usepackage{graphicx,grffile}
\makeatletter
\def\maxwidth{\ifdim\Gin@nat@width>\linewidth\linewidth\else\Gin@nat@width\fi}
\def\maxheight{\ifdim\Gin@nat@height>\textheight\textheight\else\Gin@nat@height\fi}
\makeatother
% Scale images if necessary, so that they will not overflow the page
% margins by default, and it is still possible to overwrite the defaults
% using explicit options in \includegraphics[width, height, ...]{}
\setkeys{Gin}{width=\maxwidth,height=\maxheight,keepaspectratio}
% Set default figure placement to htbp
\makeatletter
\def\fps@figure{htbp}
\makeatother
\setlength{\emergencystretch}{3em} % prevent overfull lines
\providecommand{\tightlist}{%
  \setlength{\itemsep}{0pt}\setlength{\parskip}{0pt}}
\setcounter{secnumdepth}{-\maxdimen} % remove section numbering

\title{tidyverse style guide tutorial}
\author{Grace Skinner}
\date{19/May/2022}

\begin{document}
\maketitle

\href{https://style.tidyverse.org/}{Tidyverse Style Guide Website}

\hypertarget{intro}{%
\section{Intro}\label{intro}}

\begin{itemize}
\tightlist
\item
  Style guide provides consistency, making code easier to write because
  you need to make fewer decisions
\item
  Have installed styler add-in, which allows you to restyle selected
  text/files/projects
\end{itemize}

\hypertarget{files}{%
\section{Files}\label{files}}

\hypertarget{names}{%
\subsubsection{Names}\label{names}}

\begin{itemize}
\tightlist
\item
  Meaningful
\item
  End in .R
\item
  Stick to numbers, letter, and \_ (don't use special characters)
\item
  If files should be run in a particular order, prefix them with
  numbers.

  \begin{itemize}
  \tightlist
  \item
    If it seems likely you'll have more than 10 files, left pad with
    zero
  \item
    Later on, if missed some files, just rename all, rather than using
    2a, 2b etc.
  \end{itemize}
\item
  File names that are all lower case are best due to differences between
  operating systems
\end{itemize}

\hypertarget{internal-structure}{%
\subsubsection{Internal structure}\label{internal-structure}}

\begin{itemize}
\tightlist
\item
  Use commented lines of - and = to break up your file into easily
  readable chunks.
\item
  If your script uses add-on packages, load them all at once at the very
  beginning of the file. This is more transparent than sprinkling
  library() calls throughout your code
\end{itemize}

\begin{Shaded}
\begin{Highlighting}[]
\CommentTok{# Load data ---------------------------------------------------------------}
\end{Highlighting}
\end{Shaded}

\hypertarget{syntax}{%
\section{Syntax}\label{syntax}}

\hypertarget{object-names}{%
\subsubsection{Object names}\label{object-names}}

\begin{itemize}
\tightlist
\item
  Variable and function names should use only lowercase letters,
  numbers, and \_. Use underscores (so called snake case) to separate
  words within a name.
\item
  Generally, variable names should be nouns and function names should be
  verbs. Strive for names that are concise and meaningful (this is not
  easy!).
\end{itemize}

\hypertarget{spacing}{%
\subsubsection{Spacing}\label{spacing}}

\begin{itemize}
\tightlist
\item
  Always put a space after a comma, never before, just like in regular
  English.
\item
  Do not put spaces inside or outside parentheses for regular function
  calls.
\item
  Place a space before and after () when used with if, for, or while.
\item
  Place a space after () used for function arguments
\item
  The embracing operator, \{\{ \}\}, should always have inner spaces to
  help emphasise its special behaviour
\item
  Most infix operators (==, +, -, \textless-, etc.) should always be
  surrounded by spaces

  \begin{itemize}
  \tightlist
  \item
    There are a few exceptions, which should never be surrounded by
    spaces:
  \item
    The operators with high precedence: ::, :::, \$, @, {[}, {[}{[},
    \^{}, unary -, unary +, and :.
  \end{itemize}
\end{itemize}

\hypertarget{commas}{%
\subsubsection{Commas}\label{commas}}

\textbf{Good}

\begin{Shaded}
\begin{Highlighting}[]
\CommentTok{# x[, 1]}
\end{Highlighting}
\end{Shaded}

\textbf{Bad}

\begin{Shaded}
\begin{Highlighting}[]
\CommentTok{# x[,1]}
\CommentTok{# x[ ,1]}
\CommentTok{# x[ , 1]}
\end{Highlighting}
\end{Shaded}

\hypertarget{parentheses}{%
\subsubsection{Parentheses}\label{parentheses}}

\textbf{Good}

\begin{Shaded}
\begin{Highlighting}[]
\CommentTok{# mean(x, na.rm = TRUE)}
\end{Highlighting}
\end{Shaded}

\textbf{Bad}

\begin{Shaded}
\begin{Highlighting}[]
\CommentTok{# mean (x, na.rm = TRUE)}
\CommentTok{# mean( x, na.rm = TRUE )}
\end{Highlighting}
\end{Shaded}

\textbf{Good}

\begin{Shaded}
\begin{Highlighting}[]
\CommentTok{# if (debug) \{}
\CommentTok{#   show(x)}
\CommentTok{# \}}
\end{Highlighting}
\end{Shaded}

\textbf{Bad}

\begin{Shaded}
\begin{Highlighting}[]
\CommentTok{# if(debug)\{}
\CommentTok{#   show(x)}
\CommentTok{# \}}
\end{Highlighting}
\end{Shaded}

\textbf{Good}

\begin{Shaded}
\begin{Highlighting}[]
\CommentTok{# function(x) \{\}}
\end{Highlighting}
\end{Shaded}

\textbf{Bad}

\begin{Shaded}
\begin{Highlighting}[]
\CommentTok{# function (x) \{\}}
\CommentTok{# function(x)\{\}}
\end{Highlighting}
\end{Shaded}

\hypertarget{embracing-operator}{%
\subsubsection{Embracing operator}\label{embracing-operator}}

\textbf{Good}

\begin{Shaded}
\begin{Highlighting}[]
\CommentTok{# max_by <- function(data, var, by) \{}
\CommentTok{#   data %>%}
\CommentTok{#     group_by(\{\{ by \}\}) %>%}
\CommentTok{#     summarise(maximum = max(\{\{ var \}\}, na.rm = TRUE))}
\CommentTok{# \}}
\end{Highlighting}
\end{Shaded}

\textbf{Bad}

\begin{Shaded}
\begin{Highlighting}[]
\CommentTok{# max_by <- function(data, var, by) \{}
\CommentTok{#   data %>%}
\CommentTok{#     group_by(\{\{by\}\}) %>%}
\CommentTok{#     summarise(maximum = max(\{\{var\}\}, na.rm = TRUE))}
\CommentTok{# \}}
\end{Highlighting}
\end{Shaded}

\hypertarget{infix-operators}{%
\subsubsection{Infix operators}\label{infix-operators}}

\textbf{Good}

\begin{Shaded}
\begin{Highlighting}[]
\CommentTok{# height <- (feet * 12) + inches}
\CommentTok{# mean(x, na.rm = TRUE)}
\end{Highlighting}
\end{Shaded}

\textbf{Bad}

\begin{Shaded}
\begin{Highlighting}[]
\CommentTok{# height<-feet*12+inches}
\CommentTok{# mean(x, na.rm=TRUE)}
\end{Highlighting}
\end{Shaded}

\textbf{Good}

\begin{Shaded}
\begin{Highlighting}[]
\CommentTok{# sqrt(x^2 + y^2)}
\CommentTok{# df$z}
\CommentTok{# x <- 1:10}
\end{Highlighting}
\end{Shaded}

\textbf{Bad}

\begin{Shaded}
\begin{Highlighting}[]
\CommentTok{# sqrt(x ^ 2 + y ^ 2)}
\CommentTok{# df $ z}
\CommentTok{# x <- 1 : 10}
\end{Highlighting}
\end{Shaded}

\hypertarget{function-calls}{%
\subsubsection{Function calls}\label{function-calls}}

\begin{itemize}
\tightlist
\item
  When you call a function, you typically omit the names of data
  arguments, because they are used so commonly.
\item
  If you override the default value of an argument, use the full name
\item
  Avoid assignment in function calls
\end{itemize}

\textbf{Good}

\begin{Shaded}
\begin{Highlighting}[]
\CommentTok{# mean(1:10, na.rm = TRUE)}
\end{Highlighting}
\end{Shaded}

\textbf{Bad}

\begin{Shaded}
\begin{Highlighting}[]
\CommentTok{# mean(x = 1:10, , FALSE)}
\CommentTok{# mean(, TRUE, x = c(1:10, NA))}
\end{Highlighting}
\end{Shaded}

\hypertarget{control-flow}{%
\subsubsection{Control flow}\label{control-flow}}

\begin{itemize}
\tightlist
\item
  Curly braces, \{\}, define the most important hierarchy of R code. To
  make this hierarchy easy to see:

  \begin{itemize}
  \tightlist
  \item
    \{ should be the last character on the line. Related code (e.g., an
    if clause, a function declaration, a trailing comma, \ldots) must be
    on the same line as the opening brace.
  \item
    The contents should be indented by two spaces.
  \item
    \} should be the first character on the line.
  \end{itemize}
\item
  If used, else should be on the same line as \}.
\end{itemize}

\textbf{Good}

\begin{Shaded}
\begin{Highlighting}[]
\CommentTok{# if (y < 0 && debug) \{}
\CommentTok{#   message("y is negative")}
\CommentTok{# \}}
\end{Highlighting}
\end{Shaded}

\textbf{Bad}

\begin{Shaded}
\begin{Highlighting}[]
\CommentTok{# if (y < 0 && debug) \{}
\CommentTok{# message("Y is negative")}
\CommentTok{# \}}
\CommentTok{# }
\CommentTok{# if (y == 0)}
\CommentTok{# \{}
\CommentTok{#     if (x > 0) \{}
\CommentTok{#       log(x)}
\CommentTok{#     \} else \{}
\CommentTok{#   message("x is negative or zero")}
\CommentTok{#     \}}
\CommentTok{# \} else \{ y ^ x \}}
\end{Highlighting}
\end{Shaded}

\hypertarget{long-lines}{%
\subsubsection{Long lines}\label{long-lines}}

\begin{itemize}
\tightlist
\item
  Strive to limit your code to 80 characters per line. This fits
  comfortably on a printed page with a reasonably sized font. If you
  find yourself running out of room, this is a good indication that you
  should encapsulate some of the work in a separate function.
\end{itemize}

\hypertarget{semi-colons}{%
\subsubsection{Semi-colons}\label{semi-colons}}

\begin{itemize}
\tightlist
\item
  Don't put ; at the end of a line, and don't use ; to put multiple
  commands on one line.
\end{itemize}

\hypertarget{assignment}{%
\subsubsection{Assignment}\label{assignment}}

\begin{itemize}
\tightlist
\item
  Use \textless-, not =, for assignment.
\end{itemize}

\hypertarget{data}{%
\subsubsection{Data}\label{data}}

\begin{itemize}
\tightlist
\item
  Use ", not ', for quoting text.
\item
  The only exception is when the text already contains double quotes and
  no single quotes.
\end{itemize}

\textbf{Good}

\begin{Shaded}
\begin{Highlighting}[]
\CommentTok{# "Text"}
\CommentTok{# 'Text with "quotes"'}
\CommentTok{# '<a href="http://style.tidyverse.org">A link</a>'}
\end{Highlighting}
\end{Shaded}

\textbf{Bad}

\begin{Shaded}
\begin{Highlighting}[]
\CommentTok{# 'Text'}
\CommentTok{# 'Text with "double" and \textbackslash{}'single\textbackslash{}' quotes'}
\end{Highlighting}
\end{Shaded}

\hypertarget{logical-vectors}{%
\subsubsection{Logical vectors}\label{logical-vectors}}

\begin{itemize}
\tightlist
\item
  Prefer TRUE and FALSE over T and F.
\end{itemize}

\hypertarget{comments}{%
\subsubsection{Comments}\label{comments}}

\begin{itemize}
\tightlist
\item
  Each line of a comment should begin with the comment symbol and a
  single space: \#
\item
  If you discover that you have more comments than code, consider
  switching to R Markdown.
\end{itemize}

\hypertarget{functions}{%
\section{Functions}\label{functions}}

\hypertarget{naming}{%
\subsubsection{Naming}\label{naming}}

\begin{itemize}
\tightlist
\item
  Strive to use verbs for function names
\end{itemize}

\hypertarget{long-lines-1}{%
\subsubsection{Long lines}\label{long-lines-1}}

\begin{itemize}
\tightlist
\item
  Function-indent: place each argument on its own line, and indent to
  match the opening ( of function:
\end{itemize}

\begin{Shaded}
\begin{Highlighting}[]
\CommentTok{# long_function_name <- function(a = "a long argument",}
\CommentTok{#                                b = "another argument",}
\CommentTok{#                                c = "another long argument") \{}
\CommentTok{#   # As usual code is indented by two spaces.}
\CommentTok{# \}}
\end{Highlighting}
\end{Shaded}

\begin{itemize}
\tightlist
\item
  OR Double-indent: Place each argument of its own double indented line.
\end{itemize}

\begin{Shaded}
\begin{Highlighting}[]
\CommentTok{# long_function_name <- function(}
\CommentTok{#     a = "a long argument",}
\CommentTok{#     b = "another argument",}
\CommentTok{#     c = "another long argument") \{}
\CommentTok{#   # As usual code is indented by two spaces.}
\CommentTok{# \}}
\end{Highlighting}
\end{Shaded}

\begin{itemize}
\tightlist
\item
  Function indent is preferred when it fits
\end{itemize}

\hypertarget{return}{%
\subsubsection{return()}\label{return}}

\begin{itemize}
\tightlist
\item
  Only use return() for early returns.
\item
  ?!?! Otherwise, rely on R to return the result of the last evaluated
  expression. ?!?!
\item
  Return statements should always be on their own line
\end{itemize}

\textbf{Good}

\begin{Shaded}
\begin{Highlighting}[]
\CommentTok{# find_abs <- function(x) \{}
\CommentTok{#   if (x > 0) \{}
\CommentTok{#     return(x)}
\CommentTok{#   \}}
\CommentTok{#   x * -1}
\CommentTok{# \}}
\CommentTok{# }
\CommentTok{# add_two <- function(x, y) \{}
\CommentTok{#   x + y}
\CommentTok{# \}}
\end{Highlighting}
\end{Shaded}

\textbf{Bad}

\begin{Shaded}
\begin{Highlighting}[]
\CommentTok{# add_two <- function(x, y) \{}
\CommentTok{#   return(x + y)}
\CommentTok{# \}}
\end{Highlighting}
\end{Shaded}

\hypertarget{comments-1}{%
\subsubsection{Comments}\label{comments-1}}

\begin{itemize}
\tightlist
\item
  Use comments to explain the ``why'' not the ``what'' or ``how''.
\item
  Each line of a comment should begin with the comment symbol and a
  single space: \# .
\item
  Comments should be in sentence case, and only end with a full stop if
  they contain at least two sentences
\end{itemize}

\hypertarget{pipes}{%
\section{Pipes}\label{pipes}}

\begin{itemize}
\tightlist
\item
  \%\textgreater\% should always have a space before it, and should
  usually be followed by a new line. After the first step, each line
  should be indented by two spaces
\end{itemize}

\textbf{Good}

\begin{Shaded}
\begin{Highlighting}[]
\CommentTok{# iris %>%}
\CommentTok{#   group_by(Species) %>%}
\CommentTok{#   summarize_if(is.numeric, mean) %>%}
\CommentTok{#   ungroup() %>%}
\CommentTok{#   gather(measure, value, -Species) %>%}
\CommentTok{#   arrange(value)}
\end{Highlighting}
\end{Shaded}

\textbf{Bad}

\begin{Shaded}
\begin{Highlighting}[]
\CommentTok{# iris %>% group_by(Species) %>% summarize_all(mean) %>%}
\CommentTok{# ungroup %>% gather(measure, value, -Species) %>%}
\CommentTok{# arrange(value)}
\end{Highlighting}
\end{Shaded}

\begin{itemize}
\tightlist
\item
  If the arguments to a function don't all fit on one line, put each
  argument on its own line and indent
\end{itemize}

\begin{Shaded}
\begin{Highlighting}[]
\CommentTok{# iris %>%}
\CommentTok{#   group_by(Species) %>%}
\CommentTok{#   summarise(}
\CommentTok{#     Sepal.Length = mean(Sepal.Length),}
\CommentTok{#     Sepal.Width = mean(Sepal.Width),}
\CommentTok{#     Species = n_distinct(Species)}
\CommentTok{#   )}
\end{Highlighting}
\end{Shaded}

\begin{itemize}
\tightlist
\item
  Assignment - a few alternative options but I like this one best:
\end{itemize}

\begin{Shaded}
\begin{Highlighting}[]
\CommentTok{# iris_long <- iris %>%}
\CommentTok{#   gather(measure, value, -Species) %>%}
\CommentTok{#   arrange(-value)}
\end{Highlighting}
\end{Shaded}

\hypertarget{ggplot2}{%
\section{ggplot2}\label{ggplot2}}

\begin{itemize}
\tightlist
\item
  `+' should always have a space before it, and should be followed by a
  new line. This is true even if your plot has only two layers. After
  the first step, each line should be indented by two spaces.
\item
  If you are creating a ggplot off of a dplyr pipeline, there should
  only be one level of indentation.
\end{itemize}

\textbf{Good}

\begin{Shaded}
\begin{Highlighting}[]
\CommentTok{# iris %>%}
\CommentTok{#   filter(Species == "setosa") %>%}
\CommentTok{#   ggplot(aes(x = Sepal.Width, y = Sepal.Length)) +}
\CommentTok{#   geom_point()}
\end{Highlighting}
\end{Shaded}

\textbf{Bad}

\begin{Shaded}
\begin{Highlighting}[]
\CommentTok{# iris %>%}
\CommentTok{#   filter(Species == "setosa") %>%}
\CommentTok{#   ggplot(aes(x = Sepal.Width, y = Sepal.Length)) +}
\CommentTok{#     geom_point()}
\end{Highlighting}
\end{Shaded}

\textbf{Bad}

\begin{Shaded}
\begin{Highlighting}[]
\CommentTok{# iris %>%}
\CommentTok{#   filter(Species == "setosa") %>%}
\CommentTok{#   ggplot(aes(x = Sepal.Width, y = Sepal.Length)) + geom_point()}
\end{Highlighting}
\end{Shaded}

\begin{itemize}
\tightlist
\item
  If the arguments to a ggplot2 layer don't all fit on one line, put
  each argument on its own line and indent
\end{itemize}

\textbf{Good}

\begin{Shaded}
\begin{Highlighting}[]
\CommentTok{# ggplot(aes(x = Sepal.Width, y = Sepal.Length, color = Species)) +}
\CommentTok{#   geom_point() +}
\CommentTok{#   labs(}
\CommentTok{#     x = "Sepal width, in cm",}
\CommentTok{#     y = "Sepal length, in cm",}
\CommentTok{#     title = "Sepal length vs. width of irises"}
\CommentTok{#   ) }
\end{Highlighting}
\end{Shaded}

\textbf{Bad}

\begin{Shaded}
\begin{Highlighting}[]
\CommentTok{# ggplot(aes(x = Sepal.Width, y = Sepal.Length, color = Species)) +}
\CommentTok{#   geom_point() +}
\CommentTok{#   labs(x = "Sepal width, in cm", y = "Sepal length, in cm", title = "Sepal length vs. width of irises") }
\end{Highlighting}
\end{Shaded}

\begin{itemize}
\tightlist
\item
  ggplot2 allows you to do data manipulation, such as filtering or
  slicing, within the data argument. Avoid this, and instead do the data
  manipulation in a pipeline before starting plotting.
\end{itemize}

\hypertarget{making-packages}{%
\section{Making packages}\label{making-packages}}

\hypertarget{files-1}{%
\subsubsection{Files}\label{files-1}}

\begin{itemize}
\tightlist
\item
  If a file contains a single function, give the file the same name as
  the function.
\item
  If a file contains multiple related functions, give it a concise, but
  evocative name.
\end{itemize}

\hypertarget{documentation}{%
\subsubsection{Documentation}\label{documentation}}

\begin{itemize}
\tightlist
\item
  Use the first line of your function documentation to provide a concise
  title that describes the function, dataset, or class. Titles should
  use sentence case but not end with a full stop
\item
  Always indent with one space after \#'
\item
  For most tags, like @param, @seealso and @return, the text should be a
  sentence, starting with a capital letter and ending with a full stop
\end{itemize}

\hypertarget{error-messages}{%
\subsubsection{Error messages}\label{error-messages}}

\begin{itemize}
\tightlist
\item
  If the cause of the problem is clear, use ``must'' - clear cut causes
  typically involve incorrect types or lengths.
\item
  If you cannot state what was expected, use ``can't''
\item
  The problem statement should use sentence case and end with a full
  stop.
\item
  Use simple sentences layed out in a bullet list of ℹ and ✖ elements
\item
  Do your best to reveal the location, name, and/or content of the
  troublesome component
\item
  If the list of issues might be long, make sure to truncate to only
  show the first few
\item
  If the source of the error is clear and common, you may want to
  provide a hint as to how to fix it. If UTF-8 is available, prefix with
  ℹ. Hints should always end in a question mark.
\end{itemize}

\begin{Shaded}
\begin{Highlighting}[]
\CommentTok{# dplyr::filter(iris, Species = "setosa")}
\CommentTok{# #> Error: Filter specifications must be named.}
\CommentTok{# #> ℹ Did you mean `Species == "setosa"`?}
\CommentTok{# }
\CommentTok{# ggplot2::ggplot(ggplot2::aes())}
\CommentTok{# #> Error: Can't plot data with class "uneval". }
\CommentTok{# #> ℹ Did you accidentally provide the results of aes() to the `data` argument?}
\end{Highlighting}
\end{Shaded}

\begin{itemize}
\tightlist
\item
  Good hints are difficult to write because, as above, you want to avoid
  steering users in the wrong direction. Generally, I avoid writing a
  hint unless the problem is common, and you can easily find a common
  pattern of incorrect usage
\item
  Errors should be written in sentence case, and should end in a full
  stop
\item
  If you can detect multiple problems, list up to five. This allows the
  user to fix multiple problems in a single pass without being
  overwhelmed by many errors that may have the same source
\end{itemize}

\hypertarget{news}{%
\subsubsection{News}\label{news}}

\begin{itemize}
\tightlist
\item
  Each user-facing change to a package should be accompanied by a bullet
  in NEWS.md
\item
  Strive to place the name of the function as close to the beginning of
  the bullet as possible
\end{itemize}

\textbf{Good}

\begin{Shaded}
\begin{Highlighting}[]
\CommentTok{# * `ggsave()` now uses full argument names to avoid partial match warning (#2355).}
\end{Highlighting}
\end{Shaded}

\textbf{Bad}

\begin{Shaded}
\begin{Highlighting}[]
\CommentTok{# * Fixed partial argument matches in `ggsave()` (#2355).}
\end{Highlighting}
\end{Shaded}

\begin{itemize}
\tightlist
\item
  Lines should be wrapped to 80 characters, and each bullet should end
  in a full stop.\#
\item
  Frame bullets positively (i.e.~what now happens, not what used to
  happen), and use the present tense.
\end{itemize}

\end{document}
